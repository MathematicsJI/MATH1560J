\section{Limits}
\begin{frame}
	\frametitle{"Rough" Definition of a Limit}
	Suppose $f(x)$ is defined when $x$ is near the number $a$. (This means that $f$ is defined on some open interval that contains $a$, \alert{except possibly} at $a$ itself.)\\
	Then we write
	\begin{center}
		$\lim\limits_{\textit{x} \to a}f(x) = L$
	\end{center}
	and say
	\begin{center}
		"the limit if $f(x)$, as $x$ approaches $a$, equals $L$"
	\end{center}
	if we can make the values of $f(x)$ arbitrarily close to $L$ (as close to $L$ as we like) by taking $x$ to be \alert{sufficiently close to} $a$ (on either side of $a$) but \alert{not equal to} $a$.
\end{frame}
\begin{frame}
	\frametitle{One-sided Limits}
	Considering a function called \textit{Heaviside Function}\\
	\begin{center}
		\begin{equation}
			H(t)=
			\begin{cases}
				0 & t<0      \\
				1 & t \geq 0
			\end{cases}
		\end{equation}
	\end{center}
	Does $\lim\limits_{\textit{t} \to 0}H(t)$ exists?
\end{frame}
\begin{frame}
	\frametitle{One-sided Limits}
	We write
	\begin{center}
		$\lim\limits_{\textit{x} \to a^{-}}f(x) = L$
	\end{center}
	and say the left-hand limit of $f(x)$ as $x$ approaches $a$ is equal to $L$ if we can make the values of $f(x)$ arbitrarily close to $L$ by taking $x$ to be \alert{sufficiently close to} $a$ and $x$ less than $a$.\\
	\bigskip
	When calculating $\lim\limits_{\textit{x} \to a^{-}}f(x)$, we consider only $x < a$.\\
	\bigskip
	Similarly, we can get the right-hand limit of $f(x)$ as $x$ approaches $a$.
\end{frame}
\begin{frame}
	\frametitle{One-sided Limits}
	When does $\lim\limits_{\textit{x} \to a}f(x)$ exists?\\
	\bigskip
	\begin{center}
		$\lim\limits_{\textit{x} \to a}f(x) = L$\\
		\bigskip
		\alert{if and only if}\\
		\bigskip
		$\lim\limits_{\textit{x} \to a^{-}}f(x) = L$ and $\lim\limits_{\textit{x} \to a^{+}}f(x) = L$
	\end{center}
	Can we directly regard $L$ as $f(a)$?
\end{frame}
\begin{frame}

	\frametitle{Infinite Limits}
	Let $f$ be a function defined on both sides of $a$, \alert{except possibly} at $a$ itself. Then
	\begin{center}
		$\lim\limits_{\textit{x} \to a}f(x) = \infty$
	\end{center}
	means that the values of $f(x)$ can be made arbitrarily large (as large as we please) by taking $x$ \alert{sufficiently close to} $a$, but \alert{not equal to} $a$.\\

	Let $f$ be a function defined on both sides of $a$, \alert{except possibly} at $a$ itself. Then
	\begin{center}
		$\lim\limits_{\textit{x} \to a}f(x) = -\infty$
	\end{center}
	means that the values of $f(x)$ can be made arbitrarily large negative by taking $x$ \alert{sufficiently close to} $a$, but \alert{not equal to} $a$.
\end{frame}
\begin{frame}
	\frametitle{Infinite Limits}
	\alert{Warning:}\\
	$\lim\limits_{\textit{x} \to a}f(x) = (-)\infty$ does not mean that we are regarding $(-)\infty$ as a number. Nor does it mean that the limit exists!
\end{frame}
\begin{frame}
	\frametitle{Limits at Infinity}
	\begin{enumerate}
		\item Let $f$ be a function defined on some interval ($a$, $\infty$). Then
		      \begin{center}
			      $\lim\limits_{\textit{x} \to \infty}f(x) = L$
		      \end{center}
		      means that the values of $f(x)$ can be made arbitrarily close to $L$ by taking $x$ sufficiently large.
		\item Let $f$ be a function defined on some interval ($-\infty$, $a$). Then
		      \begin{center}
			      $\lim\limits_{\textit{x} \to -\infty}f(x) = L$
		      \end{center}
		      means that the values of $f(x)$ can be made arbitrarily close to $L$ by taking $x$ sufficiently large negative.
	\end{enumerate}
\end{frame}
\begin{frame}
	\frametitle{Infinite Limits at Infinity}
	\begin{enumerate}
		\item $\lim\limits_{\textit{x} \to \infty}f(x) = \infty$
		\item $\lim\limits_{\textit{x} \to \infty}f(x) = -\infty$
		\item $\lim\limits_{\textit{x} \to -\infty}f(x) = \infty$
		\item $\lim\limits_{\textit{x} \to -\infty}f(x) = -\infty$
	\end{enumerate}
\end{frame}
\begin{frame}
	\frametitle{The Limit Laws}
	Five basic laws:\\
	Suppose that $c$ is a constant and the limits
	\begin{center}
		$\lim\limits_{\textit{x} \to a}f(x)$ and $\lim\limits_{\textit{x} \to a}g(x)$
	\end{center}
	exists. Then
	\begin{enumerate}
		\item $\lim\limits_{\textit{x} \to a}[f(x)+g(x)] = \lim\limits_{\textit{x} \to a}f(x) + \lim\limits_{\textit{x} \to a}g(x)$
		\item $\lim\limits_{\textit{x} \to a}[f(x)-g(x)] = \lim\limits_{\textit{x} \to a}f(x) - \lim\limits_{\textit{x} \to a}g(x)$
		\item $\lim\limits_{\textit{x} \to a}[cf(x)] = c\lim\limits_{\textit{x} \to a}f(x)$
		\item $\lim\limits_{\textit{x} \to a}[f(x)g(x)] = \lim\limits_{\textit{x} \to a}f(x) \cdot \lim\limits_{\textit{x} \to a}g(x)$
		\item $\lim\limits_{\textit{x} \to a}\dfrac{f(x)}{g(x)} = \dfrac{\lim\limits_{\textit{x} \to a}f(x)}{\lim\limits_{\textit{x} \to a}g(x)}$\ \ \  \alert{(if $\lim\limits_{\textit{x} \to a}g(x) \neq 0$)}
	\end{enumerate}
\end{frame}
\begin{frame}
	\frametitle{The Limit Laws}
	Another six laws:
	\begin{enumerate}
		\item $\lim\limits_{\textit{x} \to a}[f(x)]^{n} = [\lim\limits_{\textit{x} \to a}f(x)]^{n}$ (where $n$ is a positive integer)
		\item $\lim\limits_{\textit{x} \to a}c = c$
		\item $\lim\limits_{\textit{x} \to a}x = a$
		\item $\lim\limits_{\textit{x} \to a}x^{n} = a^{n}$ (where $n$ is a positive integer)
		\item $\lim\limits_{\textit{x} \to a}\sqrt[n]{x} = \sqrt[n]{a}$ (where $n$ is a positive integer)\\
		      \alert{if $n$ is even, we assume that $a > 0$}
		\item $\lim\limits_{\textit{x} \to a}\sqrt[n]{f(x)} = \sqrt[n]{\lim\limits_{\textit{x} \to a}f(x)}$ (where $n$ is a positive integer)\\
		      \alert{if $n$ is even, we assume that $\lim\limits_{\textit{x} \to a}f(x) > 0$}
	\end{enumerate}
\end{frame}
\begin{frame}
	\frametitle{The Limit Laws}
	For composite functions, we have
	\begin{block}{Law for Composite Functions}
		\[\lim\limits_{\textit{x} \to a}f(g(x)) = f(\lim\limits_{\textit{x} \to a}g(x))\]

		for all the limits and functions exist.

	\end{block}

	E.g. Solve $\lim\limits_{\textit{x} \to \infty} \cos\dfrac{1}{x}$.
\end{frame}
\begin{frame}
	\frametitle{The Limit Laws}
	\alert{Warning:}\\
	The Limit Laws can't be applied to infinite limits because $(-)\infty$ is not a number!

	\textbf{即极限进行运算时,必须保证运算前的极限存在(不是未定式/无穷),而且极限为有限个}
\end{frame}
\begin{frame}
	\frametitle{The Limit Laws}
	Two additional properties of limits:
	\begin{enumerate}
		\item if $f(x) \leq g(x)$ when $x$ is near $a$ (\alert{except possibly} at $a$) and the limits of $f$ and $g$ both exists as $x$ approaches $a$, then
		      \begin{center}
			      $\lim\limits_{\textit{x} \to a}f(x) \leq \lim\limits_{\textit{x} \to a}g(x)$
		      \end{center}
		\item \alert{(The Squeeze Theorem)} if $f(x) \leq g(x) \leq h(x)$ when $x$ is near $a$ (\alert{except possibly} at $a$) and
		      \begin{center}
			      $\lim\limits_{\textit{x} \to a}f(x) = \lim\limits_{\textit{x} \to a}h(x) = L$
		      \end{center}
		      then
		      \begin{center}
			      $\lim\limits_{\textit{x} \to a}g(x) = L$
		      \end{center}
	\end{enumerate}
\end{frame}

\begin{frame}
	\frametitle{Two Important Limits}
	Be sure to keep these two limits in mind!
	\begin{enumerate}
		\item $\lim\limits_{\textit{x} \to 0}\dfrac{\sin{x}}{x} = 1$\\
		      (How to prove it? Considering $\sin{x}$, $x$ and $\tan{x}$. Then, use the squeeze theorem.)
		\item $\lim\limits_{\textit{x} \to \infty}(1 + \dfrac{1}{x})^{x} = e = 2.718281828459045\cdots$ \\
		      (Proof: https://www.zhihu.com/question/277272238)

	\end{enumerate}
	\begin{columns}[c] % The "c" option specifies centered vertical alignment while the "t" option is used for top vertical alignment
		\begin{column}{0.5\textwidth} % Left column width
			\begin{figure}
				\includegraphics[width=1\linewidth]{res/bbb.jpg}
			\end{figure}
		\end{column}
		\begin{column}{0.5\textwidth} % Right column width
			\begin{figure}
				\includegraphics[width=1\linewidth]{res/aaa.jpg}
			\end{figure}
		\end{column}
	\end{columns}
\end{frame}


\begin{frame}
	\frametitle{Two Important Limits}
	\begin{enumerate}
		\item $\lim\limits_{\textit{x} \to 0}\dfrac{\sin{x}}{x} = 1$\\
	\end{enumerate}

	\vspace*{\baselineskip} % 插入一个空行

	相关变形:
	\begin{itemize}
		\item $\lim \limits_{x \to \infty} \dfrac{\sin x}{x} = 0$ \\
		\item $\lim \limits_{f(x) \to 0} \dfrac{\sin f(x)}{f(x)} = 1$ \\
	\end{itemize}
\end{frame}

\begin{frame}
	\frametitle{Two Important Limits}
	\begin{enumerate}
		\item $\lim\limits_{\textit{x} \to \infty}(1 + \dfrac{1}{x})^{x} = e$
	\end{enumerate}

	\vspace*{\baselineskip} % 插入一个空行

	相关变形:
	\begin{columns}
		\begin{column}{0.5\textwidth}
			\begin{itemize}
				\item $\lim \limits_{x \to 0^{+}} (1+\dfrac{1}{x})^x = 1$ \\
				\item $\lim \limits_{x\to 0} (1+x)^{\frac{1}{x}} = e$ \\
				\item $\lim \limits_{x\to +\infty} (1+x)^{\frac{1}{x}} = 1$ \\
				\item $\lim \limits_{x\to 0^{-}} (1-\dfrac{1}{x})^x = 1$ \\
				\item $\lim \limits_{x\to \infty} (1-\dfrac{1}{x})^x = \dfrac{1}{e} $
			\end{itemize}
		\end{column}

		\begin{column}{0.5\textwidth}
			If $u \to 1$, $v\to \infty$, then we have
			\[
				\mathbf{\lim u^v = "1^\infty" = e^{\lim v(u-1)}}
			\]

			也就是幂指函数(后面会讲)
		\end{column}
	\end{columns}
\end{frame}


% \colortheme{blue!50!black}

% \begin{frame}
% 	\frametitle{The Precise Definition of a Limit}
% 	Let $f$ be a function defined on some open interval that contains the number $a$, \alert{except possibly} at $a$ itself. Then we say that the limit if $f(x)$ as $x$ approaches $a$ is $L$, and we write
% 	\begin{center}
% 		$\lim\limits_{\textit{x} \to a}f(x) = L$
% 	\end{center}
% 	if for \alert{every} number $\varepsilon > 0$, there is a number $\delta > 0$ such that
% 	\begin{center}
% 		if\ \ \ $0 < |x - a| < \delta$\ \ \ then\ \ \ $|f(x) - L| < \varepsilon$
% 	\end{center}
% \end{frame}
% \begin{frame}
% 	\frametitle{The Precise Definition of a Limit}
% 	Left-hand limits:
% 	\begin{center}
% 		$\lim\limits_{\textit{x} \to a^{-}}f(x) = L$
% 	\end{center}
% 	if for \alert{every} number $\varepsilon > 0$, there is a number $\delta > 0$ such that
% 	\begin{center}
% 		if\ \ \ $a - \delta < x < a$\ \ \ then\ \ \ $|f(x) - L| < \varepsilon$
% 	\end{center}
% 	Right-hand limits:
% 	\begin{center}
% 		$\lim\limits_{\textit{x} \to a^{+}}f(x) = L$
% 	\end{center}
% 	if for \alert{every} number $\varepsilon > 0$, there is a number $\delta > 0$ such that
% 	\begin{center}
% 		if\ \ \ $a < x < a + \delta$\ \ \ then\ \ \ $|f(x) - L| < \varepsilon$
% 	\end{center}
% \end{frame}
% \begin{frame}
% 	\frametitle{The Precise Definition of a Limit}
% 	Infinite limits:\\
% 	\begin{enumerate}
% 		\item Let $f$ be a function defined on some open interval that contains the number $a$, \alert{except possibly} at $a$ itself. Then
% 		      \begin{center}
% 			      $\lim\limits_{\textit{x} \to a}f(x) = \infty$
% 		      \end{center}
% 		      means that for \alert{every positive} number $M$, there is a number $\delta > 0$ such that
% 		      \begin{center}
% 			      if\ \ \ $0 < |x - a| < \delta$\ \ \ then\ \ \ $f(x) > M$
% 		      \end{center}
% 		\item Let $f$ be a function defined on some open interval that contains the number $a$, \alert{except possibly} at $a$ itself. Then
% 		      \begin{center}
% 			      $\lim\limits_{\textit{x} \to a}f(x) = -\infty$
% 		      \end{center}
% 		      means that for \alert{every negative} number $N$, there is a number $\delta > 0$ such that
% 		      \begin{center}
% 			      if\ \ \ $0 < |x - a| < \delta$\ \ \ then\ \ \ $f(x) < N$
% 		      \end{center}
% 	\end{enumerate}
% \end{frame}
% \begin{frame}
% 	\frametitle{The Precise Definition of a Limit}
% 	Limits at Infinity:\\
% 	\begin{enumerate}
% 		\item Let $f$ be a function defined on some interval ($a$, $\infty$). Then
% 		      \begin{center}
% 			      $\lim\limits_{\textit{x} \to \infty}f(x) = L$
% 		      \end{center}
% 		      means that for every $\varepsilon > 0$, there is a corresponding number $N$ such that
% 		      \begin{center}
% 			      if\ \ \ $x > N$\ \ \ then\ \ \ $|f(x) - L| < \varepsilon$
% 		      \end{center}
% 		\item Let $f$ be a function defined on some interval ($-\infty$, $a$). Then
% 		      \begin{center}
% 			      $\lim\limits_{\textit{x} \to -\infty}f(x) = L$
% 		      \end{center}
% 		      means that for every $\varepsilon > 0$, there is a corresponding number $N$ such that
% 		      \begin{center}
% 			      if\ \ \ $x < N$\ \ \ then\ \ \ $|f(x) - L| < \varepsilon$
% 		      \end{center}
% 	\end{enumerate}
% \end{frame}

\colortheme{red!95!black}
\begin{frame}
	\frametitle{常见极限计算方法}
	\textbf{常见极限计算方法:}

	\begin{enumerate}
		\item 四则运算与函数性质\\
		\item 导数定义
		\item \alert{两个重要极限}\\
		\item \alert{洛必达}\\
		\item \textit{等价无穷小}\\
		\item \textit{泰勒展开(了解即可)}
	\end{enumerate}

\end{frame}

\colortheme{green!30!black}

\begin{frame}
	\frametitle{Exercise 1}
	\textbf{四则运算与函数性质}

	有理式函数极限:\\

	对有理式$F(x) = \dfrac{p(x)}{q(x)}=\dfrac{a_0 + a_1 x_1 + \cdots + a_m x^m }{b_0 + b_1 x_1 + \cdots + b_n x^n}$, 有

	$\lim \limits_{x \to \infty} F(x) = \begin{cases}
			\dfrac{a_m}{b_n} & m = n \\
			0                & m < n \\
			\infty           & m > n
		\end{cases}
	$

	1. 计算下列极限:
	\begin{enumerate}
		\item $\lim\limits_{\textit{x} \to 0}\dfrac{4x^{3}-2x^{2}+x}{2x+3x^{2}}$
		\item $\lim\limits_{\textit{x} \to \infty}\dfrac{3x^3 + 4x^2 +2 }{7x^3+5x^2-3}$
		\item $\lim\limits_{\textit{x} \to \infty}(1+\dfrac{1}{x})(2-\dfrac{1}{x^{2}})$
	\end{enumerate}
\end{frame}

\begin{frame}
	\frametitle{Exercise 1}
	\textbf{四则运算与函数性质}

	\vspace*{1em}

	2. 设$\lim \limits_{x \to -1} \dfrac{x^3-ax^2-x+4}{x+1}$存在极限值且为$m$, 试求$a$和$m$的值。

	\vspace*{2em}

	\textbf{有根号:分子有理化或根式整体换元}

	\vspace*{1em}

	3. 计算下列极限:
	\begin{enumerate}
		\item  $\lim\limits_{\textit{x} \to 0}\dfrac{-1+\sqrt[n]{x+1}}{x}$
		\item  $\lim\limits_{\textit{x} \to -4}\dfrac{\sqrt{9+x^{2}}-5}{x+4}$
		\item  $\lim \limits_{\textit{x} \to -\infty}(\sqrt{x^2+2x}+x)$
		\item  $\lim \limits_{\textit{x} \to 0}\dfrac{\sqrt{1+\tan^2{x}}-\sqrt{1+\sin^2{x}}}{(5^x-1)\arctan^3{x}}$

	\end{enumerate}
\end{frame}



\begin{frame}
	\frametitle{Exercise 2}
	\textbf{导数定义}

	\vspace*{1em}

	计算下列极限:
	\begin{enumerate}
		\item $\lim\limits_{\textit{\alert{h}} \to 0}\dfrac{(x+h)^{3}-x^{3}}{h}$\\
		      (What if $\lim\limits_{\textit{\alert{h}} \to 0}\dfrac{(x+h)^{n}-x^{n}}{h}$?)

	\end{enumerate}
	\vspace*{1em}
\end{frame}
% \colortheme{blue!50!black}

\begin{frame}
	\frametitle{Exercise 2}
	\textbf{\alert{两个重要极限}}

	\vspace*{1em}

	计算下列极限:
	\begin{enumerate}
		\item $\lim\limits_{\textit{x} \to 0}\dfrac{\sin{\omega x}}{x}$
		\item $\lim\limits_{\textit{x} \to \infty}(1-\dfrac{1}{x})^{kx}$\ \ \ ($k$ is a positive integer)
		\item $\lim\limits_{\textit{x} \to \frac{\pi}{2}}(\sin{x})^{\tan{x}}$
		\item $\lim \limits_{x \to 0^{+}} (\cos \sqrt{x})^{\frac{\pi}{x}}$
		\item $\lim \limits_{x \to \infty}\left(\dfrac{x+3}{x+6}\right)^{\frac{x-1}{2}}$
	\end{enumerate}
	\vspace*{1em}
\end{frame}




\colortheme{pink!90!black}



\begin{frame}{L'Hospital's Rule}
	\begin{block}{L'Hospital's Rule}
		Suppose $f$ and $g$ are differentiable and $g^\prime(x)\neq0$ on an open interval $I$ that contains $a$ (except possibly at $a$). Suppose that\\
		$$
			\lim_{x\rightarrow a}f(x)=0\quad \text{and}\quad \lim_{x\rightarrow a}g(x)=0
		$$
		or that
		$$
			\lim_{x\rightarrow a}f(x)=\pm\infty\quad \text{and}\quad \lim_{x\rightarrow a}g(x)=\pm\infty
		$$
		(In other words, we have an indeterminate form of type $\frac{0}{0}$ or $\infty/\infty$.) Then\\
		$$
			\lim_{x\rightarrow a}\frac{f(x)}{g(x)}=\lim_{x\rightarrow a}\frac{f^\prime(x)}{g^\prime(x)}
		$$
		if the limit on the right side exists (or is $\infty$ or $-\infty$).
	\end{block}
\end{frame}






\begin{frame}
	\frametitle{Exercise 13}
	\framesubtitle{l'Hôpital's rule}
	\alert{Warning}: Always judge whether l'Hôpital's rule can be applied before you use it, and \alert{don't neglect those basic methods of finding the limit}.\\
	\bigskip
	Evaluate the following limits:
	\begin{enumerate}
		\item $\lim\limits_{\textit{x} \to 0^{+}}x^{\sin{x}}$
		\item $\lim\limits_{\textit{x} \to 1}\dfrac{\ln x}{x-1}$
		\item $\lim\limits_{\textit{x} \to \infty}x^{3}e^{-x^{2}}$
		\item $\lim\limits_{\textit{x} \to \infty}[x - x^{2}\ln{\dfrac{x + 1}{x}}]$\\
		      \begin{figure}
			      \includegraphics[width=0.1\linewidth]{res/troll.png}
		      \end{figure}
		\item \begin{center}
			      $\lim\limits_{\textit{x} \to \infty}\dfrac{x}{\sqrt{x^{2} + 1}}$
		      \end{center}
	\end{enumerate}
	\ref{ans:exercise13}
\end{frame}



\begin{frame}{Equivalent Infinitesimal}
	\textbf{Tip:} You're highly recommended to remember this part!
	$$
		\begin{aligned}
			 & \textbf{When } x \rightarrow 0          \\
			 & a^{x}-1 \sim x \ln a                    \\
			 & \arcsin (a) x \sim \sin (a) x \sim(a) x \\
			 & \arctan (a) x \sim \tan (a) x \sim(a) x \\
			 & \ln (1+x) \sim x                        \\
			 & \sqrt{1+x}-\sqrt{1-x} \sim x            \\
			 & (1+a x)^{b}-1 \sim a b x                \\
			 & \sqrt[b]{1+a x}-1 \sim \frac{a}{b} x    \\
			 & 1-\cos x \sim \frac{x^{2}}{2}           \\
			 & x-\ln (1+x) \sim \frac{x^{2}}{2}        \\
		\end{aligned}
	$$
\end{frame}


\colortheme{blue!50!black}


\begin{frame}{Equivalent Infinitesimal}
	$$
		\begin{aligned}
			 & \textbf{When } x \rightarrow 0     \\
			 & \tan x-\sin x \sim \frac{x^{3}}{2} \\
			 & \tan x-x \sim \frac{x^{3}}{3}      \\
			 & x-\arctan x \sim \frac{x^{3}}{3}   \\
			 & x-\sin x \sim \frac{x^{3}}{6}      \\
			 & \arcsin x-x \sim \frac{x^{3}}{6}   \\
		\end{aligned}
	$$
\end{frame}





\begin{frame}{Equivalent Infinitesimal}
	\begin{block}{Example: Solve the limit}
		$$
			\begin{aligned}
				\lim _{x \rightarrow 0} \frac{\ln (1+4 x)}{\sin (3 x)} & =\lim _{x \rightarrow 0} \frac{\ln (1+4 x)}{\sin (3 x)} \lim _{x \rightarrow 0} \frac{4 x}{\ln (1+4 x)} \lim _{x \rightarrow 0} \frac{\sin (3 x)}{3 x} \\&=\lim _{x \rightarrow 0} \frac{4 x}{3 x}=4 / 3
			\end{aligned}
		$$
	\end{block}
	For more exercise regarding to equivalent infinitesimal, please refer to Worksheet 1.

	\vspace*{1em}

	\textbf{注意:加减法中不可使用部分的等价无穷小代换!只有乘除形式的才可以代换}


\end{frame}

\begin{frame}{Equivalent Infinitesimal}
	\textbf{注意:加减法中不可使用部分的等价无穷小代换!}
	\begin{block}{Example: Solve the limit}
		\begin{itemize}
			\item $\lim \limits_{x \to 0} \dfrac{x-\sin x\cos x}{x^3}$
			\item $\lim \limits_{x \to \infty} \dfrac{(1+\dfrac{1}{x})^{x^2}}{e^x}$
		\end{itemize}
	\end{block}




\end{frame}

\colortheme{blue!50!black}


\begin{frame}{Taylor Expansion}
	\begin{block}{Definition}
		Taylor expansion around $x=x_0$:\\
		$f(x)=f(x_0)+\sum\limits_{i=1}^n\frac{f^{(i)}(x)}{i!}(x-x_0)^i+R_n$, where $R_n=o[(x-x_0)^n]$\\
	\end{block}

	It simulates a function around a point with a polynomial function.
\end{frame}



\begin{frame}{Taylor Expansion}

	Taylor expansion of some polynomials when x is around 0:
	\begin{enumerate}
		\item $e^x=1+x+\frac{x^2}{2}+\frac{x^3}{6}+o(x^3)$
		\item $ln(1+x)=x-\frac{x^2}{2}+\frac{x^3}{3}+o(x^3)$
		\item $sinx=x-\frac{x^3}{6}+\frac{x^5}{120}+o(x^5)$
		\item $cosx=1-\frac{x^2}{2}+\frac{x^4}{24}+o(x^4)$
		\item $tanx=x+\frac{x^3}{3}+\frac{2x^5}{15}+o(x^5)$
	\end{enumerate}
	\begin{block}{Tip}
		\footnotesize
		$o(x^n)\ means\ the\ order\ of\ the\ polynomial\ is\ \alert{larger}\ than\ n;$\\
		$O(x^n)\ means\ the\ order\ of\ the\ polynomial\ is\ \alert{larger\ than\ or\ equal\ to}\ n.$
		\normalsize
	\end{block}
\end{frame}


\begin{frame}{Taylor Expansion}


	The transformation of Taylor Expansion:
	\begin{block}{Example}
		The Taylor expansion of $e^{x^2}$ around\ x=0:\\
		$e^{x^2}=1+x^2+\frac{(x^2)^2}{2}+\frac{(x^2)^3}{6}+o((x^2)^3)=1+x^2+\frac{x^4}{2}+\frac{x^6}{6}+o(x^6)$
	\end{block}


\end{frame}


% \begin{frame}{Exercise 11}
%     \begin{block}{}
%         \begin{enumerate}
%             \item Calculate The Taylor expansion of:\\
%                   \begin{enumerate}
%                       \item $e^{sinx}$ around\ x=0 (below degree 4)
%                       \item $ln(2+x)$ around\ x=-1 (below degree 4)
%                   \end{enumerate}
%             \item Calculate the limit $$\lim\limits_{x\to 0}\frac{ln(e^{sinx}+sinx)-ln(e^{tanx}+tanx)}{x^2\cdot tanx}$$
%         \end{enumerate}

%     \end{block}

% \end{frame}


% \begin{frame}{Exercise 11}
%     \small
%     Solution:\\
%     (1)\\
%     \begin{enumerate}
%         \item $e^{sinx}=1+(x-\dfrac{1}{6}x^3+O(x^5))+\dfrac{1}{2}(x-\dfrac{1}{6}x^3+O(x^5))^2+\dfrac{1}{6}(x+O(x^3))^3+\dfrac{1}{24}(x+O(x^3))^4=1+x+\dfrac{x^2}{2}-\dfrac{x^4}{8}+O(x^5)$
%         \item $ln(2+x)=ln(1+(1+x))=ln(1+x)=(1+x)-\frac{(1+x)^2}{2}+\frac{(1+x)^3}{3}-\frac{(1+x)^4}{4}+O((1+x)^5)$
%     \end{enumerate}
% \end{frame}


% \begin{frame}{Exercise 11}
%     Solution:\\
%     (2)\\
%     The denominator's Taylor expansion is $x^2tanx=x^3+O(x^5)$, thus for the numerator we can ignore elements of degree 4 or higher.\\
%     \begin{enumerate}
%         \item $e^x+x=1+2x+\dfrac{x^2}{2}+\dfrac{x^3}{6}+O(x^4)$
%         \item $ln(e^x+x)=2x+\dfrac{x^2}{2}+\dfrac{x^3}{6}-\dfrac{1}{2}(2x+\dfrac{x^2}{2}+\dfrac{x^3}{6})^2+\dfrac{1}{3}(2x+\dfrac{x^2}{2}+\dfrac{x^3}{6})^3+O(x^4)=2x-\dfrac{3}{2}x^2-\dfrac{1}{2}x^3+O(x^4)$
%         \item $ln(e^{sinx}+sinx)=2x-\dfrac{1}{3}x^3-\dfrac{3}{2}x^2-\dfrac{1}{2}x^3+O(x^4)$
%         \item$ln(e^{tanx}+tanx)=2x+\dfrac{2}{3}x^3-\dfrac{3}{2}x^2-\dfrac{1}{2}x^3+O(x^4)$
%     \end{enumerate}
%     \begin{center}
%         $ans=\dfrac{-x^3}{x^3}=-1$
%     \end{center}
%     \normalsize

% \end{frame}


\colortheme{green!30!black}


\begin{frame}
	\frametitle{Conclusions}
	Some methods to calculate the limits:
	\begin{enumerate}
		\item Use those limit laws directly
		\item Exchange the order of functions and limit symbols based on the continuity of composite function. (Will be mentioned later)
		\item Do factorization, denominator rationalization or numerator rationalization.
		\item If a factor approaching zero is find in the denominator, try to eliminate it.
		\item Translate the formula into the form of "two important limits"
		\item \alert{The method to solve those formulas having the form of $u(x)^{v(x)}$ will be discussed at a deeper level after the differentiation and l'Hôpital's rule are taught.}
	\end{enumerate}
\end{frame}






\begin{frame}{Exercise Answer 1}
	\begin{enumerate}
		\item $\lim\limits_{\textit{x} \to 0}\dfrac{4x^{3}-2x^{2}+x}{2x+3x^{2}}=\lim\limits_{\textit{x} \to 0}\dfrac{4x^{2}-2x+1}{2+3x}$=0.5
		\item $\lim\limits_{\textit{x} \to 0}\dfrac{\dfrac{1}{3+x}-\dfrac{1}{3}}{x}=\lim\limits_{\textit{x} \to 0}\dfrac{3-3-x}{3x(3+x)}=\lim\limits_{\textit{x} \to 0}\dfrac{-1}{3(3+x)}=-\dfrac{1}{9}$
		\item $\lim\limits_{\textit{x} \to \infty}(1+\dfrac{1}{x})(2-\dfrac{1}{x^{2}})=\lim\limits_{\textit{x} \to \infty}(1+\dfrac{1}{x})\cdot\lim\limits_{\textit{x} \to \infty}(2-\dfrac{1}{x^{2}})=1\cdot2=2$
		\item Let $t^{n}-1:=x$, $\lim\limits_{\textit{x} \to 0}\dfrac{-1+\sqrt[n]{x+1}}{x}=\lim\limits_{\textit{t} \to 1}\dfrac{t-1}{t^{n}-1}=\lim\limits_{\textit{t} \to 1}\dfrac{1}{1+t+t^2+\cdots+t^{n-1}}=\dfrac{1}{n}$
	\end{enumerate}
	\label{ans:exercise1}
\end{frame}
\colortheme{blue!50!black}
\begin{frame}{Exercise Answer 2}
	\small
	\begin{enumerate}
		\item $\lim\limits_{\textit{x} \to -4}\dfrac{\sqrt{9+x^{2}}-5}{x+4}=\lim\limits_{\textit{x} \to -4}\dfrac{x-4}{\sqrt{9+x^{2}}+5}=-\dfrac{4}{5}$
		\item $\lim\limits_{\textit{\alert{h}} \to 0}\dfrac{(x+h)^{n}-x^{n}}{h}=\lim\limits_{\textit{\alert{h}} \to 0}nx^{n-1}+h\cdot(\cdots)=nx^{n-1}$
		\item $\lim\limits_{\textit{x} \to 0}\dfrac{\sin{\omega x}}{x}=\omega\cdot\lim\limits_{\textit{x} \to 0}\dfrac{\sin{\omega x}}{\omega x}=\omega$
		\item $\lim\limits_{\textit{x} \to \infty}(1-\dfrac{1}{x})^{kx}=\lim\limits_{\textit{x} \to \infty}(1+\dfrac{1}{-x})^{(-x)(-k)}=e^{-k}$
		\item $\lim\limits_{\textit{x} \to \dfrac{\pi}{2}}(\sin{x})^{\tan{x}}=\lim\limits_{\textit{x} \to \dfrac{\pi}{2}}(1+(\sin{x}-1))^{\dfrac{1}{\sin{x}-1}\cdot(\sin{x}-1)\tan{x}}=e^{\lim\limits_{\textit{x} \to \dfrac{\pi}{2}}(\sin{x}-1)\tan{x}}=e^{\lim\limits_{\textit{x} \to \dfrac{\pi}{2}}\dfrac{(\sin{x}-1)\sin{x}}{\sqrt{1-\sin^2x}}}=e^{-\lim\limits_{\textit{x} \to \dfrac{\pi}{2}}\dfrac{\sqrt{1-\sin{x}}\sin{x}}{\sqrt{1+\sin x}}}=e^0=1$
	\end{enumerate}
	\normalsize
	\label{ans:exercise2}
\end{frame}

\begin{frame}{Solutions to Exercise 13}
	Solution:
	\begin{enumerate}
		\item 1
		\item 1
		\item $\lim\limits_{\textit{x} \to \infty}\dfrac{x^3}{e^{x^2}} = \lim\limits_{\textit{x} \to \infty} \dfrac{3x^2}{2xe^{x^2}} = \lim\limits_{\textit{x} \to \infty} \dfrac{6x}{(2 + 4x^2)e^{x^2}} =  \lim\limits_{\textit{x} \to \infty} \dfrac{6}{12x + 8x^2} = 0$
		\item $ u = \dfrac{1}{x}$


		      $\lim\limits_{\textit{u} \to 0} \dfrac{u - ln(u+1)}{u^2} = \lim\limits_{\textit{u} \to 0} \dfrac{1 - \dfrac{1}{u+1}}{2u} = \lim\limits_{\textit{u} \to 0} \dfrac{\dfrac{1}{(u+1)^2}}{2} = \dfrac{1}{2}$
		\item $\lim\limits_{\textit{x} \to \infty} \dfrac{\sqrt{x^2 + 1}}{x} = 1$
	\end{enumerate}
	\label{ans:exercise13}
\end{frame}

\colortheme{green!30!black}
\begin{frame}{References}
	\frametitle{References}
	[1] Huang, Yucheng. VV156\_RC2.pdf. 2021.\\
	\bigskip
	[2] Cai, Runze. Chapter01.pdf. 2021.\\
	\bigskip
	[3] Department of mathematics, Tongji University. Advanced Mathematics (7th Edition). 2014.\\
	\bigskip
	[4] James Stewart. Calculus (7th Edition). 2014.\\
	\bigskip
	[5] Department of mathematics, Tongji University. Learning Guidance of Advanced Mathematics (7th Edition). 2014.\\
	\bigskip
	[6]Zhou,Yishen.RC2. 2022.
\end{frame}